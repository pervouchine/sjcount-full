\documentclass{article}
\usepackage{tikz}
\usepackage{array}
\usepackage{graphicx}
\usepackage{lscape}
\usepackage{amsmath}
\usepackage{multirow}
\usepackage{hyperref}
\usepackage[affil-it]{authblk}

\DeclareMathOperator{\SJ}{SJ}

\renewcommand{\baselinestretch}{1.2}

\begin{document}

\title{Fast quantification of splice junctions from RNA-seq data by {\em sjcount} v3.0}

\author{Dmitri D. Pervouchine\thanks{email: dp@crg.eu}}\affil{Centre for Genomic Regulation (CRG), Barcelona, Spain}

\date{\today}
\maketitle

\section{Synopsis}

The purpose of {\em sjcount} is to provide a fast utility for counting splice junctions in BAM files. 
It is the annotation-agnostic version of \href{https://github.com/pervouchine/bam2ssj}{bam2ssj}. 
This document describes the version {\bf v3.0} of {\em sjcount}. The older versions of {\em sjcount}
(v1.0, v2.0) is also included in the package inder the name {\em deprecated}.

\section{Changes since v2.0}
The has been a substrantial change between v2.0 and v3.0.
\begin{enumerate}
\item The utility now counts and reports reads with multisplits
\item Accordingly, the output format has changed to account for multisplits
\item A simplier and more efficient data structure is now used to store and parse multisplits
\item Test rountines are now added to check the quality and integrity of the output as compared to the output of 
a perl script with easily controlled syntax
\end{enumerate}

\section{Installation and usage}
See README.md file for installation instructions. The program {\em sjcount} is used from the command line with the following keys
\begin{verbatim}
sjcount -bam bam_file [-ssj junctions_output] [-ssc boundary_output]
       [-read1 0|1] [-read2 0|1] [-unstranded] [-nbins number_of_bins]
       [-lim number_of_lines] [-quiet]
\end{verbatim}
where
\begin{itemize}
\item {\bf bam\_file} is a sorted input BAM file with a header
\item {\bf junctions\_output} is the output file with junction counts
\item {\bf boundary\_output} is the output file with boundary counts
\item {\bf read1} 0/1, reverse complement read1 no/yes (default=no)
\item {\bf read2} 0/1, reverse complement read2 no/yes (default=no)
\item {\bf unstranded}, force strand=0
\item {\bf nbins} number of offset bins, (default=1)
\item {\bf maxnh} the max value of the NH tag, (default=none)
\item {\bf lim} stop after reading these many lines, (default=no limit)
\item {\bf quiet} -- suppress verbose output {\bf NOTE: use -quiet if you redirect stderr to a file!}
\end{itemize}

The output consists of two files. First, a tab-delimited file containing multi-split counts is produced as follows
\begin{verbatim}
chr1_100_200_+            1       34      1
chr1_100_200_+            1       36      1
chr1_100_200_+            1       37      6
chr1_100_200_+            1       38      3
chr1_100_200_300_400_+    2       49      1
chr1_100_400_+            1       33      1
...                       ...     ...     ...
\end{verbatim}
where the first column contains the coordinates of the split alignment (including multi-splits, see below). The second 
column contains the numer of splits. The third column contains the offset sefined as the distance within the short read 
sequence of the latest split (defined below). The last column is the respective count, i.e., the number of split-mapped 
reads with the given combination of alignment coordinates and offset. 

For instance, {\tt chr1\_100\_200\_+} denotes an alignment that was split once between positions 100 and 200 on the '+' 
strand, while {\tt chr1\_100\_200\_300\_400\_+} denotes an alignment that was split twice, first between positions 100 
and 200, and then between positions 300 and 400. The coordinates are 1-based and always refer to terminal {\em exonic} 
nucleotides. The strand is denoted by '+' and '-' for stranded data or by '.' for unstranded data.

The second output is also a tab-delimited file which contains the counts of read alignments that {\em overlap} exon 
boundries (exon boundries are defined by splice junctions in the previous file). In this version all alignments that 
overlap an exon boundary by at least one nucleotide are counted (in older versions only continuous alignments were counted. 
This second file is optional and is needed to compute the completness of splicing index~\cite{pmid23172860, pmid22955974}.

\section{Method}
By definition, we say that we observe a {\em splice junction} each time we see an 'N' symbol in the CIGAR attribute 
of the alignment. If the CIGAR attribute contains several N's, then we have a {\em multi-split} or $n$-split, where $n$ 
is the number of N's in CIGAR. In this terms, each 1-split defines one splice junction while each $n$-split defines
$n$ splice junctions. 

Each multi-split is counted according to the number of splits so that, for example, the 
alignment {\tt chr1\_100\_200\_300\_400\_500\_600\_+} is counted once as a 3-split, two times 
as a 2-split ({\tt chr1\_100\_200\_300\_400\_+} and {\tt chr1\_300\_400\_500\_600\_+}), and three times 
as single-split ({\tt chr1\_100\_200\_+}, {\tt chr\_300\_400\_+}, and {\tt chr1\_500\_600\_+}). 
The positions of splits are decided entirely by the mapper which produced the alignment.

As an example, consider the multi-split alignment shown in Figure~\ref{fig::01} below. 
%
\begin{figure}[h]
\footnotesize
\begin{verbatim}
       10        20        30        40        50        60        70        80
       |         |         |         |         |         |         |         |
       12345678 9012345678901234567890123456789012345678901234567890123456789012

chr1   AGTCTAGG*GACGGCATAGGAGGTGAGCATTTGTGTACGCAGATCTACAAAACATGTGTCACGGATAGGATCG
Query     CTAGGAGACGG**TAGGAG....................ATCTA*AAAACAT.............GATa
                            |<-----   SJ1  ----->|           |<--- SJ2 --->|
\end{verbatim}
The corresponding SAM line is:
\begin{verbatim}
Query   123   chr1  14    255    5M1I5M2D6M20N5M1D7M13N3M1S 1234 
\end{verbatim}
\caption{An example alignment and its CIGAR attribute\label{fig::01}}
\end{figure}
%
In the output file it will be counted in three lines: in {\tt chr1\_31\_52\_+} as having 1 split, 
in {\tt chr1\_64\_78\_+} as having 1 split, and in {\tt chr1\_31\_52\_64\_78\_+} as having 2 splits.
One may want to subset the output to regular splice junctions by requiring the second column be 
equal to one.

Artifacts may arise from combining counts that come from different starting positions of the alignment. We 
define the {\em offset} to be the distance ({\em in the query sequence!}) from the first alignment position 
to the corresponding 'N'. For instance, the junction $\SJ_1$ in Figure~\ref{fig::01} has offset $17$, while
the junction $\SJ_1$ has offset $29$. The offset of the multi-split is defined to be the offest of it's last 
N, i.e., $29$ in this case. Since the offset is defined as a position in the query sequence, its value
cannot exceed the read length.

Some offsets may give artifactually large read counts corresponding to PCR artefacts~\cite{pmid22537040}. 
In Figure~\ref{fig::02} we show six split reads supporting the same splice junction with offsets~14 (Q1), 
12 (Q2--Q4), and~8 (Q5--Q6). Note that offsets appear decreasing when sequentially processing lines a sorted 
BAM file.
%
\begin{figure}[h]
\footnotesize
\begin{verbatim}
       10       20        30        40        50        60        70        80
       |        |         |         |         |         |         |         |
       123456789012345678901234567890123456789012345678901234567890123456789012

Ref    AGTCTAGGGACGGCATAGGAGGTGAGCATTTGTGTACGCAGATCTACAAAACATGTGTCACGGATAGGATCG

Q1            GGACGGCATAGGAG....................ATCT      
Q2              ACGGCATAGGAG....................ATCTAC    
Q3              ACGGCATAGGAG....................ATCTAC    
Q4              ACGGCATAGGAG....................ATCTAC    
Q5                  CATAGGAG....................ATCTACAAAA
Q6                  CATAGGAG....................ATCTACAAAA
\end{verbatim}
\caption{Split-mapped reads support the same splice junction with different offsets\label{fig::02}}
\end{figure}

The quantification of abundance is done as follows. We initialize and keep $nbins$ separate counters 
for each $n$-split. For each instance of $n$-split, we increment the counter corresponding to its 
offset. If the offset is larger than or equal to $nbins$ then it is  set to be equal to $nbins-1$.

For example, in the default settings we have $nbins=1$. This means that the bin number will be $1-1=0$ 
for all supporting reads, regardless of their offset ($t=14$ for Q1, $t=12$ for Q2--Q4, and $t=8$ for 
Q5--Q6 in Figure~\ref{fig::02}). Therefore, there is only one counter to increment, and the result will 
be so called "collapsed" counts. The output corresponding to Figure~\ref{fig::02} will then be
\begin{verbatim}
Ref_31_52_+     1       0       6
\end{verbatim}

By contrast, if we set $nbins$ equal to read length, there will be a separate counter for each offset 
and the output corresponding to Figure~\ref{fig::02} will be
\begin{verbatim}
Ref_31_52_+     1       8       2
Ref_31_52_+     1       12      3
Ref_31_52_+     1       14      1
\end{verbatim}

One single number is usually reported for each splice junction as an endpoint. Normally, the user wants 
to know how many reads aligned to a certain split regardless of the offset. This number is equal to the 
sum of counts for the given alignment over all values of offset. In other words, the total number of counts
is obtained from offset-specific counts by aggregation using the function $f(x_1,\dots,x_n) = x_1+\dots+x_n$.

Offset-specific counts are quite useful when aggregating by using different functions. For example, the number 
of {\em staggered} counts is the result of aggregation using  $f(x_1,\dots,x_n) = \theta(x_1)+\dots+\theta(x_n)$, 
where $\theta(x)=1$ for $x>0$ and $\theta(x)=0$ for $x\le0$. Another useful function is {\em entropy}, which
is obtained from the offset-specific counts by aggregation with
$$f(x_1,\dots,x_n) = \log_2(\sum\limits_{i=1}^nx_i) - \frac{\sum\limits_{i=1}^nx_i\log_2(x_i)}{\sum\limits_{i=1}^nx_i}.$$
The entropy and the number of staggered reads can be used to filter out artefactual read counts. Note that 
{\em sjcount} only reports offset-specific counts, while the aggregation is left to the user. 

\bibliography{sjcount}
\bibliographystyle{abbrv}

\end{document}

