\documentclass{article}
\usepackage{tikz}
\usepackage{array}
\usepackage{graphicx}
\usepackage{lscape}
\usepackage{amsmath}
\usepackage{multirow}
\usepackage{hyperref}

\DeclareMathOperator{\SJ}{SJ}

\begin{document}
\title{Fast quantification of splice junctions by {\em sjcount}}
\author{Dmitri D. Pervouchine}
\date{\today}
\maketitle

\section{Synopsis}

The purpose of {\em sjcount} is to provide a fast method for quantification of splice junctions from BAM files. It is an annotation-agnostic version of 
\href{https://github.com/pervouchine/bam2ssj}{bam2ssj}.


\section{Installation and usage}

See README.md file for installation instructions. {\em sjcount} is used with the following keys

{\bf NOTE} that you need to install samtools.
Samtools package Version: 0.1.18-dev (r982:313) is compartible (and likely so are oder versions).

\begin{verbatim}
sjcount -bam bam_file [-ssj junctions_output] [-ssc boundaries_output] [-log log_file] 
       [-maxlen max_intron_length] [-minlen min_intron_length] [-margin length] 
       [-read1 0|1] [-read2 0|1] [-nbins number_of_bins] [-binsize bin_size] 
       [-lim number_of_lines] [-quiet]
\end{verbatim}
where
\begin{itemize}
\item {\bf maxlen} upper limit on intron length, 0 = no limit (default=0)
\item {\bf minlen} lower limit on intron length, 0 = no limit (default=0)
\item {\bf margin} length, see below, (default=0)
\item {\bf read1} 0/1, reverse complement read1 no/yes (default=no)
\item {\bf read2} 0/1, reverse complement read2 no/yes (default=no)
\item {\bf binsize} size of the overhang bin, (default=$\infty$)
\item {\bf nbins} number of overhang bins, (default=1)
\item {\bf lim} nreads stop after nreads, (default=no limit)
\item {\bf quiet} -- suppress verbose output
\end{itemize}

The input to {\em sjcount} is a sorted BAM file with a header. The output consists of two files.
First, a tab-delimited .ssj file is produced. It contains counts of splice junctions, taking into 
account the strand information and also start and stop positions. Second, it produces a tab-delimited
.ssc file containing counts of continuous (non-split reads) which *overlap* splice sites defined by 
splice junctions. The second file is optional and is used to compute the completness of splicing 
index~\cite{}.

\section{Definitions}

By definition, we will say that we observe a splice junction whenever we encounter 'N' symbol in the CIGAR 
attribute of a SAM alignment. For instance, the alignment shown in Figure~\ref{fig::01} below gives rise to 
two splice junctions, $\SJ_1$ and $\SJ_2$.
%
\begin{figure}[h]
\footnotesize
\begin{verbatim}
       10        20        30        40        50        60        70        80
       |         |         |         |         |         |         |         |
       12345678 9012345678901234567890123456789012345678901234567890123456789012

Ref    AGTCTAGG*GACGGCATAGGAGGTGAGCATTTGTGTACGCAGATCTACAAAACATGTGTCACGGATAGGATCG
Query     CTAGGAGACGG**TAGGAG....................ATCTA*AAAACAT.............GATa
                            |<-----   SJ1  ----->|           |<--- SJ2 --->|
\end{verbatim}

The corresponding SAM line is:
\begin{verbatim}
Query   123   Ref   14    255    5M1I5M2D6M20N5M1D7M13N3M1S 1234 
\end{verbatim}
\caption{An example alignment and its CIGAR attribute\label{fig::01}}
\end{figure}
%
We have a convention that coordinates of splice junctions always use terminal exonic nucleotides, i.e., $\SJ_1$ is 
$\rm Ref\_31\_52$ and $\SJ2$ is $\rm Ref\_64\_78$. We denote the length of the intron by $l(\SJ)$, i.e. $l(\SJ_1)=52-31-1=20$ 
and $l(\SJ_2)=78-64-1=13$. Intron length is always equal to the corresponding 'N' number in the CIGAR attribute.

With each splice junctions we associate four numbers: $m_u$ ($m_d$) --- the number of \underline{m}atching nucleotides 
immediately upstream (downstream) of the junction, and $v_u$ ($v_d$) --- the length in the reference of the aligned region,
also called o\underline{v}erhang, including M/I/D operations and located immediately upstream (downstream) of the junction.
in Figure~\ref{fig::01} we have $m_u(\SJ_1)=6$, $m_d(\SJ_1)=5$, $v_u(\SJ_1)=31-14+1=18$, $v_d(\SJ_1)=64-52+1=13$ and
$m_u(\SJ_2)=7$, $m_d(\SJ_2)=3$, $v_u(\SJ_2)=64-52+1=13$, $v_d(\SJ_2)=80-78+1=3$.

For each splice junction we require that
\begin{enumerate}
\item $l(\SJ)\ge \rm\bf minlen$ and $l(\SJ)\le \rm\bf maxlen$
\item $m_u\ge \rm\bf margin$ and $m_d\ge \rm\bf margin$
\end{enumerate}

In addition to the coordinates of a junction, the upstream overhang $v_u(\SJ)$ is also taken into account to 
distinguish staggered and non-staggered reads (Figure~\ref{fig::02}).
%
\begin{figure}[h]
\footnotesize
\begin{verbatim}
       10       20        30        40        50        60        70        80
       |        |         |         |         |         |         |         |
       123456789012345678901234567890123456789012345678901234567890123456789012

Ref    AGTCTAGGGACGGCATAGGAGGTGAGCATTTGTGTACGCAGATCTACAAAACATGTGTCACGGATAGGATCG

Q1            GGACGGCATAGGAG....................ATCT      
Q2              ACGGCATAGGAG....................ATCTAC    
Q3              ACGGCATAGGAG....................ATCTAC    
Q4              ACGGCATAGGAG....................ATCTAC    
Q5                  CATAGGAG....................ATCTACAAAA
Q6                  CATAGGAG....................ATCTACAAAA
\end{verbatim}
\caption{Counting convention\label{fig::02}}
\end{figure}
%

This is done as follows. For each instance of a splice junction we increment the corresponding counter for the
bin defined by $d=floor(v_u/binsize)$. For example, in the default settings $binsize=+\infty$. Then, $d=0$
for all supporting reads, regardless of their overhang ($v_u=14$ for Q1, $v_u=12$ for Q2--4, and $v_u=8$ for Q5--6). 
Therefore, there is only one counter to increment, and the result will be the ``collapsed'' counts. The output 
corresponding to Figure~\ref{fig::02} will be
\begin{verbatim}
Ref     31      52      1       0       6
\end{verbatim}

By contrast, to count split reads taking into account the overhang information, one should set $binsize=1$ 
(and specify $nbins$ because the program doesn't know the range of possible offsets). There will be a separate 
counter for each offset and the output corresponding to Figure~\ref{fig::02} will then look like
\begin{verbatim}
Ref     31      52      1       8        2
Ref     31      52      1       12       3
Ref     31      52      1       14       1
\end{verbatim}



\end{document}

