\documentclass{article}
\usepackage{tikz}
\usepackage{array}
\usepackage{graphicx}
\usepackage{lscape}
\usepackage{amsmath}
\usepackage{multirow}
\usepackage{hyperref}
\usepackage[affil-it]{authblk}

\DeclareMathOperator{\SJ}{SJ}

\renewcommand{\baselinestretch}{1.2}

\begin{document}

\title{Quantification of Splice Junctions in RNA-seq Data by {\em sjcount} v3.0}

\author{Dmitri D. Pervouchine\thanks{email: dp@crg.eu}}\affil{Centre for Genomic Regulation (CRG), Barcelona, Spain}

\date{\today}
\maketitle
\tableofcontents

\section{Synopsis}

The purpose of {\em sjcount} is to provide a fast utility for counting splice junctions in BAM files. 
It is the annotation-agnostic version of \href{https://github.com/pervouchine/bam2ssj}{bam2ssj}. 
This document describes the version {\bf v3.0} of {\em sjcount}. The older versions of {\em sjcount}
(v1.0, v2.0) is also included in the package, but their use is deprecated.

\section{Changes since previous version}
The has been a substantial change between v2.0 and v3.0.
\begin{enumerate}
\item The utility now counts and reports all reads with multi-splits (see definitions below)
\item Accordingly, the output format has changed to accomidate multi-splits. One of the columns reports 
the number of splits (1 for splice junctions) and another column reports the coordinates of the multi-split (separated by "\_"). 
\item A simpler and more efficient data structure is now used to store and parse multi-splits.
\item Test routines are now added to check the quality and integrity of the output as compared to the output of 
a pearl script which does the same counting job but has much easier syntax
\end{enumerate}

\section{Installation and usage}
See README.md file for installation instructions. The program {\em sjcount} is used from the command line with the following keys
\begin{verbatim}
sjcount_v3 -bam bam_file [-ssj junctions_output] [-ssc boundary_output]
       [-read1 0|1] [-read2 0|1] [-unstranded] [-nbins number_of_bins]
       [-lim number_of_lines] [-quiet]
\end{verbatim}
where
\begin{itemize}
\item {\bf bam\_file} is a sorted input BAM file with a header
\item {\bf junctions\_output} is the output file with junction counts
\item {\bf boundary\_output} is the output file with boundary counts
\item {\bf read1} 0/1, reverse complement read1 no/yes (default=yes)
\item {\bf read2} 0/1, reverse complement read2 no/yes (default=no)
\item {\bf unstranded}, force strand=0
\item {\bf continuous}, no mismatches when counting overlaps of splice boundaries
\item {\bf nbins} number of offset bins, (default=1)
\item {\bf maxnh} the max value of the NH tag, (default=none)
\item {\bf lim} stop after reading these many lines, (default=no limit)
\item {\bf quiet} -- suppress verbose output {\bf NOTE: use -quiet if you redirect stderr to a file!}
\end{itemize}

The output consists of two files. First, a tab-delimited file containing multi-split counts is produced as follows
\begin{verbatim}
chr1_100_200_+            1       34      1
chr1_100_200_+            1       36      1
chr1_100_200_+            1       37      6
chr1_100_200_+            1       38      3
chr1_100_200_300_400_+    2       49      1
chr1_100_400_+            1       33      1
...                       ...     ...     ...
\end{verbatim}
where the first column contains the coordinates of the splits in the alignment (including multi-splits, see below). 
The second column contains the number of splits. The third column contains the {\em offset} defined as the distance within 
the short read sequence of the latest split (defined below). The last column is the respective number of counts, i.e., the 
number of split-reads with the combination of coordinates and offset specified in columns 1 and 3.

For instance, {\tt chr1\_100\_200\_+} denotes an alignment that was split once between positions 100 and 200 on the '+' 
strand (referred to as 1-split), while {\tt chr1\_100\_200\_300\_400\_+} denotes an alignment that was split twice 
(referred to as 2-split), first between positions 100 and 200, and then between positions 300 and 400. There are certain 
conventions on how many times 1-splits and 2-splits are counted (see below). The coordinates are always {\bf 1-based} 
and always refer to terminal {\em exonic} nucleotides. The rationale behind it is that in applications we often need to
find an intron adjacent to the given exon, and adding/subtracting one nucleotide depending on the strand becomes a headache 
each time. The strand is denoted by '+' and '-' for stranded data or by '.' for unstranded data.

The second output is also a tab-delimited file which contains the counts of read alignments that {\em overlap} exon 
boundaries (exon boundaries are the positions of splits, as defined above). All alignments that overlap an exon boundary 
by at least one nucleotide are counted (although see -continuous flag below). This second file is optional and is needed 
to compute the completeness of splicing index~\cite{pmid23172860, pmid22955974}.

\section{Method}
\subsection{Definitions}
By definition, we say that we observe a {\em splice junction} each time we see an 'N' symbol in the CIGAR attribute 
of the alignment. If the CIGAR attribute contains several N's, then we have a {\em multi-split} or $n$-split, where $n$ 
is the number of N's in CIGAR. In this terms, each 1-split defines one splice junction while each $n$-split defines
$n$ splice junctions. 

Each multi-split is counted according to the number of splits so that, for example, the 
alignment {\tt chr1\_100\_200\_300\_400\_500\_600\_+} is counted once as a 3-split, two times 
as a 2-split ({\tt chr1\_100\_200\_300\_400\_+} and {\tt chr1\_300\_400\_500\_600\_+}), and three times 
as single-split ({\tt chr1\_100\_200\_+}, {\tt chr\_300\_400\_+}, and {\tt chr1\_500\_600\_+}). The rationale behind 
it is that the user must be able to use {\tt awk} to select $n$-split of interest. Therefore, all the components of,
for example, a 2-split also have to be counted as 1-splits.

Note that {\bf the positions of splits are decided entirely by the mapper which produced the alignment}. The mission of 
{\tt sjcount}  is to count (i) the abundancies of $n$-splits and (ii) the abundancies of reads that overlap split 
positions in $n$-splits. Nothing else.

As an example, consider the multi-split alignment shown in Figure~\ref{fig::01} below. 
%
\begin{figure}[h]
\footnotesize
\begin{verbatim}
       10        20        30        40        50        60        70        80
       |         |         |         |         |         |         |         |
       12345678 9012345678901234567890123456789012345678901234567890123456789012

chr1   AGTCTAGG*GACGGCATAGGAGGTGAGCATTTGTGTACGCAGATCTACAAAACATGTGTCACGGATAGGATCG
Query     CTAGGAGACGG**TAGGAG....................ATCTA*AAAACAT.............GATa
                            |<-----   SJ1  ----->|           |<--- SJ2 --->|
\end{verbatim}
The corresponding SAM line is:
\begin{verbatim}
Query   123   chr1  14    255    5M1I5M2D6M20N5M1D7M13N3M1S 1234 
\end{verbatim}
\caption{An example alignment and its CIGAR attribute\label{fig::01}}
\end{figure}
%
In the output file it will be counted in three lines: in {\tt chr1\_31\_52\_+} as having 1 split, 
in {\tt chr1\_64\_78\_+} as having 1 split, and in {\tt chr1\_31\_52\_64\_78\_+} as having 2 splits.
One may want to subset the output to regular splice junctions by requiring the second column be 
equal to one.

\subsection{Offset-specific counts}
Artifacts may arise when combining counts that come from different starting positions of the alignment. We 
define the {\em offset} to be the distance ({\em in the query sequence!}) from the first alignment position 
to the corresponding 'N'. For instance, the junction $\SJ_1$ in Figure~\ref{fig::01} has offset $17$, while
the junction $\SJ_1$ has offset $29$. The offset of the multi-split is defined to be the offset of it's last 
N, i.e., $29$ in the case shown in figure~\ref{fig::01}. Since the offset is defined as a position in the query 
sequence, its value cannot exceed the read length.

Some offsets may give artifactually large read counts that are usually attributed to PCR 
artifacts~\cite{pmid22537040}. In Figure~\ref{fig::02} we show six split reads supporting the same splice 
junction with offsets~14 (Q1), 12 (Q2--Q4), and~8 (Q5--Q6). Note that offsets appear decreasing when sequentially 
processing lines a sorted BAM file.
%
\begin{figure}[h]
\footnotesize
\begin{verbatim}
       10       20        30        40        50        60        70        80
       |        |         |         |         |         |         |         |
       123456789012345678901234567890123456789012345678901234567890123456789012

chr1   AGTCTAGGGACGGCATAGGAGGTGAGCATTTGTGTACGCAGATCTACAAAACATGTGTCACGGATAGGATCG

Q1            GGACGGCATAGGAG....................ATCT      
Q2              ACGGCATAGGAG....................ATCTAC    
Q3              ACGGCATAGGAG....................ATCTAC    
Q4              ACGGCATAGGAG....................ATCTAC    
Q5                  CATAGGAG....................ATCTACAAAA
Q6                  CATAGGAG....................ATCTACAAAA
\end{verbatim}
\caption{Split-mapped reads support the same splice junction with different offsets\label{fig::02}}
\end{figure}

Offset-specific counts are generated as follows. We initialize and keep $nbins$ separate counters
for each $n$-split. For each instance of $n$-split, we increment the counter corresponding to its 
offset. If the offset is larger than or equal to $nbins$ then it is  set to be equal to $nbins-1$.

For example, in the default settings we have $nbins=1$. This means that the bin number will be $1-1=0$ 
for all supporting reads, regardless of their offset ($t=14$ for Q1, $t=12$ for Q2--Q4, and $t=8$ for 
Q5--Q6 in Figure~\ref{fig::02}). Therefore, there is only one counter to increment, and the result will 
be so called "collapsed" counts. The output corresponding to Figure~\ref{fig::02} will then be
\begin{verbatim}
Ref_31_52_+     1       0       6
\end{verbatim}

By contrast, if we set $nbins$ equal to read length, there will be a separate counter for each offset 
and the output corresponding to Figure~\ref{fig::02} will be
\begin{verbatim}
Ref_31_52_+     1       8       2
Ref_31_52_+     1       12      3
Ref_31_52_+     1       14      1
\end{verbatim}

\subsection{Aggregation}
One single number is usually reported for each splice junction as an endpoint. Normally, the user wants 
to know how many reads aligned to a certain split regardless of the offset. This number is equal to the 
sum of counts for the given alignment over all values of offset. In other words, the total number of counts
is obtained from offset-specific counts by aggregation using the function $f(x_1,\dots,x_n) = x_1+\dots+x_n$.

Offset-specific counts are quite useful when aggregating by using different functions. For example, the number 
of {\em staggered} counts is the result of aggregation using  $f(x_1,\dots,x_n) = \theta(x_1)+\dots+\theta(x_n)$, 
where $\theta(x)=1$ for $x>0$ and $\theta(x)=0$ for $x\le0$. Another useful function is {\em entropy}, which
is obtained from the offset-specific counts by aggregation with
$$f(x_1,\dots,x_n) = \log_2(\sum\limits_{i=1}^nx_i) - \frac{\sum\limits_{i=1}^nx_i\log_2(x_i)}{\sum\limits_{i=1}^nx_i}.$$
The entropy and the number of staggered reads can be used to filter out artefactual read counts. Note that 
{\em sjcount} only reports offset-specific counts, while the aggregation is left to the user. These 
aggregation functions are implemented in the splicing pipeline package 
\href{https://github.com/pervouchine/ipsa}{https://github.com/pervouchine/ipsa}.

\subsection{Reads overlapping exon boundaries}
By definition, we say that an alignment overlaps an exon boundary, if the terminal exonic nucleotide and the 
following first intronic nucleotide is also aligned. For example, consider exon boundaries defined by the alignment Q1
in Figure~\ref{fig::02}.
%
\begin{figure}[h]
\footnotesize
\begin{verbatim}
       10       20        30        40        50        60        70        80
       |        |         |         |         |         |         |         |
       123456789012345678901234567890123456789012345678901234567890123456789012

chr1   AGTCTAGGGACGGCATAGGAGGTGAGCATTTGTGTACGCAGATCTACAAAACATGTGTCACGGATAGGATCG

Q1            GGACGGCATAGGAG....................ATCT      
Q7             GACGGCATAGGAGGTGA
Q8              ACGGCATAGGAGGTGAG
Q9               CGGCATAGGA-GTGAGC
Q9                GGCAT.....................................TGTGTCACGGA
\end{verbatim}
\caption{Reads overlapping exon boundaries\label{fig::03}}
\end{figure}
In this example, Q7 and Q8 overlap the exon boundary at position 31, while Q1, Q9, and Q10 
do not (note that Q1 defines exon boundary, but it would be incorrect to count it as
contributing to intron retention).

\section{Benchmark}
In principle there is not too much to benchmark in {\em sjcount} because it only does the job of counting.
We nevertheless test the performance of {\em sjcount} with respect to a na{\"\i}ve counting routine 
implemented in {\tt perl}. In order to initiate benchmark, call {\tt 'make test'} in {\em sjcount} directory.
It will download a small BAM file, run {\em sjcount}, and prompt of the outputs are different. Offset is 
also defined in this case as the position of the alignned exon boundary in the query sequence. For example,
the offset of Q7 in Figure~\ref{fig::03} is equal to 13. The position of exon boundary is denoted by {\tt chr\_pos\_str}, 
e.g., {\tt chr1\_31\_+} and {\tt chr1\_52\_+} as in Figure~\ref{fig::03}. The number of splis is equal to 0.

\bibliography{sjcount}
\bibliographystyle{abbrv}

\end{document}

